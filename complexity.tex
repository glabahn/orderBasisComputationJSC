
\section{Computational Complexity}

\label{sec:complexity}

In this section, we analyze the computational complexity of \prettyref{alg:mab}.
 
\begin{lem}
\prettyref{alg:mab} computes a $\left(\mathbf{F},\sigma,\vec{s}\right)$-basis
in no more than $\log\left(n/m\right)-1$ iterations.\end{lem}
\begin{pf}
Each iteration $i$ computes a $(\bar{\mathbf{F}}^{\left(i\right)},2\delta^{\left(i\right)},\vec{s}^{\left(i\right)})$-basis.
At iteration $\log\left(n/m\right)-1$, the degree parameter is $\sigma/2$
and $(\bar{\mathbf{F}}^{\left(i\right)},2\delta^{\left(i\right)},\vec{s}^{\left(i\right)})=\left(\mathbf{F},\sigma,\vec{s}\right)$.\end{pf}
\begin{lem}
\label{lem:remainingNumberElements}If the shift $\vec{s}=\left[0,\dots,0\right]$,
then a $\left(\mathbf{F},\sigma,\vec{s}\right)_{\delta^{\left(i\right)}-1}$-basis
(or equivalently a $(\bar{\mathbf{F}}^{\left(i\right)},2\delta^{\left(i\right)},\vec{s}^{\left(i\right)})_{\delta^{\left(i\right)}-1}$-basis)
computed at iteration $i$ has at least $n-n/2^{i}$ elements, hence
at most $n/2^{i}$ elements remain to be computed. If the shift $\vec{s}$
is balanced, that is, $\max\vec{s}\in O(d)$ assuming $\min\vec{s}=0$,
then the number $n^{\left(i\right)}$ of remaining basis elements
at iteration $i$ is $O(n/2^{i})$.\end{lem}
\begin{pf}
The uniform case follows from the idea of \citet{storjohann-villard:2005}
on null space basis computation discussed in \prettyref{sub:Unbalanced-Output}.
For the balanced case, the average column degree is bounded by $cd=cm\sigma/n$
for some constant $c$. The first iteration $\lambda$ such that $\delta^{\left(\lambda\right)}$
reaches $cd$ is therefore a constant. That is, $\delta^{\left(\lambda\right)}=2^{\lambda}d\ge cd>\delta^{\left(\lambda-1\right)}$
and hence $\lambda=\left\lceil \log c\right\rceil $. By the same
argument as in the uniform case, the number of remaining basis elements
$n^{\left(i\right)}\le n/2^{i-\lambda}=2^{\lambda}(n/2^{i})\in O(n/2^{i})$
at iteration $i\ge\lambda$. For iterations $i<\lambda$, certainly
$n^{\left(i\right)}\le n<2^{\lambda}(n/2^{i})\in O(n/2^{i})$.\end{pf}
\begin{thm}
\label{thm:balancedCost}If the shift $\vec{s}$ is balanced with
$\min\left(\vec{s}\right)=0$, then \prettyref{alg:mab} computes
a $\left(\mathbf{F},\sigma,\vec{s}\right)$-basis with a cost of $O^{\sim}\left(\MM\left(n,d\right)\right)$
field operations.\end{thm}
\begin{pf}
The computational cost depends on the degree, the row dimension, and
the column dimension of the problem at each iteration. The degree
parameter $\delta^{\left(i\right)}$ is $2^{i}d$ at iteration $i$.
The number of block rows $l^{\left(i\right)}$ is $\sigma/\delta^{\left(i\right)}-1$,
which is less than $\sigma/(2^{i}d)=n/(2^{i}m)$ at iteration $i$.
The row dimension is therefore less than $n/2^{i}$ at iteration $i$.

The column dimension of interest at iteration $i$ is the column dimension
of $\hat{\mathbf{P}}_{2}^{\left(i-1\right)}$ (equivalently the column
dimension of $\bar{\mathbf{P}}_{2}^{\left(i-1\right)}$), which is
the sum of two components, $n^{\left(i-1\right)}+(l^{\left(i-1\right)}-1)m$.
The first component $n^{\left(i-1\right)}\in O(n/2^{i})$ by \prettyref{lem:remainingNumberElements}.
The second component $(l^{\left(i-1\right)}-1)m<n/2^{i-1}-m<n/2^{i-1}$
comes from the size of the identity matrix added in Storjohann's transformation.
Therefore, the overall column dimension of the problem at iteration
$i$ is $O(n/2^{i})$.

At each iteration, the four most expensive operations are the multiplications
at \prettyref{line:matrixProduct1} and \prettyref{line:matrixProduct2},
the order basis computation at \prettyref{line:orderBasisComputation},
and extracting the basis at \prettyref{line:LSP}.

The matrices $\bar{\mathbf{F}}^{\left(i\right)}$ and $\hat{\mathbf{P}}_{2}^{\left(i-1\right)}$
have degree $O(2^{i}d)$ and dimensions $O(n/2^{i})\times O\left(n\right)$
and $O\left(n\right)\times O(n/2^{i})$. The multiplication cost is
therefore $2^{i}\MM(n/2^{i},2^{i}d)$ field operations, which is in
$O^{\sim}\left(\MM\left(n,d\right)\right)$. The matrices $\hat{\mathbf{P}}_{2}^{\left(i-1\right)}$
and $\bar{\mathbf{Q}}^{\left(i\right)}$ of the second multiplication
have the same degree $O(2^{i}d)$ and dimensions $O\left(n\right)\times O(n/2^{i})$
and $O(n/2^{i})\times O(n/2^{i})$ and can also be multiplied with
a cost of $O^{\sim}\left(\MM\left(n,d\right)\right)$ field operations.
The total cost of the multiplications over $O(\log\left(n/m\right))$
iterations is therefore $O^{\sim}\left(\MM\left(n,d\right)\right)$.

The input matrix $\mathbf{G}^{\left(i\right)}=\bar{\mathbf{F}}^{\left(i\right)}\hat{\mathbf{P}}_{2}^{\left(i-1\right)}$
of the order basis computation problem at iteration $i$ has dimension
$O(n/2^{i})\times O(n/2^{i})$ and the order of the problem is $2\delta^{\left(i\right)}\in O(2^{i}d)$.
Therefore, the cost of the order basis computation at iteration $i$
is $O^{\sim}(\MM(n/2^{i},2^{i}d)\subset O^{\sim}\left(\MM\left(n,d\right)\right)$.
The total cost over $O(\log\left(n/m\right))$ iterations is then
$O^{\sim}\left(\MM\left(n,d\right)\right)$. %
\begin{comment}
not sure if this needs more explanation. 
\end{comment}
{}

Finally, extracting an order basis by LSP factorization costs $O\left(n^{\omega}\right)$,
which is dominated by the other costs. Combining the above gives $O^{\sim}(\MM\left(n,d\right))$
as the total cost of the algorithm. 
\end{pf}

